\section{Conclusions and further work}

The main objective of this Master's Thesis was to extend the knowledge behind of the allosteric interactions occurring in the MSI-1 proteins.\\

% •	Una descripción de las conclusiones del trabajo:
% ¿Una vez se han obtenido los resultados qué conclusiones se extrae?
From the results observed, it can be concluded that the allosteric behavior of MSI-1 must stem from a conformational change within the protein that alters the RNA binding region. The efficiency at which a particular fatty acid induces said conformational change is yet to be studied.\\

However, the lipid-RRM1 models hint that the smaller and more flexible the fatty acid molecule is, the more capability it may have to induce conformational changes as it can explore RRM1's fatty acid binding ``pocket'' in more depth. As a consequence, the fatty acid interacts with a larger amount of aminoacids, changing their chemical environment and forcing a rearrangement that can spread to the RNA binding site.\\

% ¿Estos resultados son los esperados? ¿O han sido sorprendentes? ¿Por qué?
The results obtained were partially expected: most allosteric proteins undergo conformational changes when bound to their different ligands. However, it was also expected to obtain successful models for each of the RNA motif mutants to understand the effect of the RNA nucleic acid sequence itself on the interaction. Unfortunately, due to the unsatisfactory models obtained, this question will remain unanswered in this work.\\

% •	Una reflexión crítica sobre la consecución de los objetivos planteados inicialmente:
% ¿Hemos alcanzado todos los objetivos? Si la respuesta es negativa, ¿por qué?
% •	Un análisis crítico del seguimiento de la planificación y metodología a lo largo del producto:
% -	¿Se ha seguido la planificación?
% -	¿La metodología prevista ha sido suficientemente adecuada?
% -	¿Ha sido necesario introducir cambios para garantizar el éxito del trabajo? ¿Por qué?
The objectives stated during the initial phases of this work differ vastly from the objectives stated and accomplished in this report. Originally, each docking simulation would have been chaperoned by molecular dynamics simulations, which would have shed additional light on the matter of the allosteric behavior of RRM1. These molecular dynamics simulations were eventually scraped out from the objectives due to lack of time caused by solving troubles in the protein-RNA docking step.\\

Nonetheless, solving those problems resulted in a collaboration with the\linebreak\texttt{LightDock} team for a public tutorial on how to use \texttt{LightDock} for protein-RNA docking, so it was worth the time.\\

% •	Las líneas de trabajo futuro que no han podido explorarse en este trabajo y han quedado pendientes.
Therefore, additional further work could be focused on molecular dynamics assays of the allosteric interactions in MSI-1, as there is expected to observe interesting conformational changes when the ligands approach and bind RRM1.\\

It would also be of interest to assess the strength of the protein-RNA interaction in such a way that it made possible the intelligent design of RNA ligands with pre-computed affinity.\\

Additionally, it would be of great benefit for the general public to extend\linebreak\texttt{LightDock}'s functionalities into lipid-protein docking, or even general small\linebreak molecule-protein docking.\\

In sum, docking simulations have a great potential of untangling molecular mysteries such as the one involving MSI-1's allosteric nature. Nonetheless, there are still some limitations as the resulting models are not always reliable. So, there is much work to do to improve the exciting field of structural bioinformatics. Again, as Richard Feynman once said: ``\textit{what I cannot create, I do not understand}''.