\section{State of the art}

% Estado del arte del tema en cuestión.
% Debería acabar mostrando por qué el trabajo es importante y aporta algo y con las hipótesis del trabajo.

\subsection{Synthetic Biology}
Synthetic Biology is a modern niche of Biology in which the characteristic ``know-how'' from engineering fields is combined with classical Molecular Biology techniques. Guided by physicist Richard Feynman's famous quote ``\textit{what I cannot create, I do not understand}'', synthetic biologists have developed a deep understanding of how cells control and regulate the processes of gene transcription, translation, metabolite biosynthesis and degradation.\\

It is ``synthetic'' in the sense that synthetic biologists engineer novel (absent in nature) genetic constructions with biological ``pieces'', which usually originate from various different organisms, in order to achieve a phenotype in a target organism that lacks said phenotype in nature. Furthermore, this design usually comes with a mathematical model based in physical first principles of how the system should behave.\\

One of the first successful constructions of this type, and most likely the most popular one, is Gardner's toggle switch \cite{gardner_2000} where they engineered a synthetic toggle switch within \textit{E. coli} by using two different promoters that repressed each other, achieving this way bistability (either one promoter is repressed, or the other). Simultaneously, Elowitz' repressilator \cite{elowitz_2000} was published, where they engineered a synthetic ``clock'' within \textit{E. coli} by chaining 3 repressive promoters that repressed one another in chain, in such a way that the genes controlled by them had periodic expression.\\

These constructions, and all the ones that followed, would not have been possible without a (by then, small) repository of genetic parts, which in turn would not have been possible without the arduous work of an innumerable amount of traditional molecular biologists that sequenced and characterized said parts.In fact, the iGEM foundation keeps a repository of standarized biological parts \cite{parts_igem} along with various data such as the assembly methods used, levels of expression, polymerase used, etc...\\

However, there is still a need for more biological parts, especially parts that are ``orthogonal'', i.e. they can co-exist and function correcly within the same organism without deviating from their theoretical behavior. 

% RNA binding proteins
\subsection{RNA-binding proteins}

% Doking
\subsection{Molecular Docking}