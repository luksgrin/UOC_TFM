\section{Conclusions and further work}


% Este capítulo debe incluir:
% •	Una descripción de las conclusiones del trabajo:
% o	¿Una vez se han obtenido los resultados qué conclusiones se extrae?
% o	¿Estos resultados son los esperados? ¿O han sido sorprendentes? ¿Por qué?
% •	Una reflexión crítica sobre la consecución de los objetivos planteados inicialmente:
% o	¿Hemos alcanzado todos los objetivos? Si la respuesta es negativa, ¿por qué?
% •	Un análisis crítico del seguimiento de la planificación y metodología a lo largo del producto:
% o	¿Se ha seguido la planificación?
% o	¿La metodología prevista ha sido suficientemente adecuada?
% o	¿Ha sido necesario introducir cambios para garantizar el éxito del trabajo? ¿Por qué?
% •	De los impactos previstos en 1.3 (ético-sociales, de sostenibilidad y de diversidad), evaluar/mencionar si se han mitigado (si eran negativos) o si se han logrado (si eran positivos). 
% •	Si han aparecido impactos no previstos en 1.3, evaluar/mencionar cómo se han mitigado (si eran negativos) o qué han aportado (si eran positivos).
% •	Las líneas de trabajo futuro que no han podido explorarse en este trabajo y han quedado pendientes.
