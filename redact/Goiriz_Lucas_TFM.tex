\documentclass[a4paper,12pt]{article}
\usepackage[numbib,nottoc,notlof]{tocbibind}
\usepackage{graphicx}
\usepackage{helvet}
\usepackage{wrapfig}
\usepackage[table]{xcolor}
\usepackage{geometry}
\usepackage{setspace}
\usepackage{float}
\usepackage{hyperref}
\usepackage{fancyhdr}
\usepackage{boldline}
\usepackage[labelfont=bf]{caption}
\usepackage{amsmath}
\usepackage{listings}
\usepackage{xcolor}
\usepackage{longtable}

\renewcommand{\familydefault}{\sfdefault}
\renewcommand{\contentsname}{Index}
\renewcommand{\headrulewidth}{0pt}
\renewcommand{\footrule}{\hbox to\headwidth{\color{uoccyan}\leaders\hrule height \footrulewidth\hfill}}
\fancypagestyle{UOCfooterpage}{
    \fancyhf{}
    \renewcommand{\footrulewidth}{0.5pt}
    \fancyfoot[L]{\color{uocblue} Máster Universitario en Bioinformática y Bioestadística\hspace{2.2cm} 02/09/2022}
    }
\def\arraystretch{1.5}
\definecolor{uoccyan}{RGB}{115, 237, 255}
\definecolor{uocblue}{RGB}{0, 0, 120}
\definecolor{uocgrey}{RGB}{230, 230, 230}
\setlength\fboxsep{1cm}
\setlength{\parindent}{0pt}
\hypersetup{
    colorlinks=true,
    linkcolor=black,
    citecolor=black,
    urlcolor=blue
} 
\pagestyle{fancy}

\definecolor{codegreen}{rgb}{0,0.6,0}
\definecolor{codegray}{rgb}{0.5,0.5,0.5}
\definecolor{codepurple}{rgb}{0.58,0,0.82}
\definecolor{backcolour}{rgb}{0.95,0.95,0.92}

\lstdefinestyle{mystyle}{
    backgroundcolor=\color{backcolour},   
    commentstyle=\color{codegreen},
    keywordstyle=\color{magenta},
    numberstyle=\tiny\color{codegray},
    stringstyle=\color{codepurple},
    basicstyle=\ttfamily\footnotesize,
    breakatwhitespace=false,         
    breaklines=true,                 
    captionpos=b,                    
    keepspaces=true,                 
    numbers=left,                    
    numbersep=5pt,                  
    showspaces=false,                
    showstringspaces=false,
    showtabs=false,                  
    tabsize=2
}
\lstset{style=mystyle}

%%%%%%%%%%%%%%%%%%%%%%%%%%%%%%%%%%%%%%%%%%%%%%%%%%%%%%%%%%%
% Customs
\newcommand{\thistitle}{Docking assays of Musashi-1 allosteric RNA-binding sites}
\newcommand{\thisauthor}{Lucas Goiriz Beltrán}
\newcommand{\thissupervisor}{Emmanuel Fajardo}
\newcommand{\thisPRA}{Emmanuel Fajardo}
\newcommand{\duedate}{12 de Enero 2023}
\newcommand{\duedateshort}{01/2023}
%%%%%%%%%%%%%%%%%%%%%%%%%%%%%%%%%%%%%%%%%%%%%%%%%%%%%%%%%%%


\begin{document}

%%%%%%%%%%%%%%%%%%%%%%%%%%%%%%%%%%%%%%%%%%%%%%%%%%%%%%%%%%%
%                       Title page                        %
%%%%%%%%%%%%%%%%%%%%%%%%%%%%%%%%%%%%%%%%%%%%%%%%%%%%%%%%%%%
\pagenumbering{gobble}
\newgeometry{left=2cm,right=1cm,top=2cm,bottom=0cm}

\colorbox{uoccyan}{\parbox{0.84\textwidth}{
    \setstretch{3}
    \vspace{1.2cm}
    \color{uocblue}\fontsize{40pt}{1cm}\textbf{\thistitle}
    \vspace{1cm}
}}

\vspace{1cm}

\begin{minipage}{\paperwidth}
    \begin{minipage}{0.35\paperwidth}
        \includegraphics[width=6cm]{assets/UOC_UB_logos}
    \end{minipage}
    \hspace{0.001\paperwidth}
    \begin{minipage}{0.45\paperwidth}
        \fontsize{24pt}{1cm}\textbf{\thisauthor}\\
        
        \vspace{0.5cm}
        
        {\fontsize{24pt}{1cm}\selectfont MU Bioinf. i Bioest.\\
        Área de trabajo final}\\
        
        \vspace{0.5cm}
        
        \fontsize{24pt}{1cm}\textbf{Nombre Tutor/a de TF}\\

        \vspace{0.1cm}

        {\fontsize{21pt}{1cm}\selectfont \thissupervisor}\\

        \vspace{0.1cm}

        \fontsize{24pt}{1cm}\textbf{Profesor/a responsable de la asignatura}\\

        \vspace{0.1cm}

        {\fontsize{21pt}{1cm}\selectfont \thisPRA}\\

        \vspace{0.5cm}

        \fontsize{21pt}{1cm}\textbf{Fecha Entrega}\\

        \vspace{0.5cm}

        {\fontsize{21pt}{1cm}\selectfont \duedate}

        \vspace{2cm}
    \end{minipage}
\end{minipage}

\pagebreak

%%%%%%%%%%%%%%%%%%%%%%%%%%%%%%%%%%%%%%%%%%%%%%%%%%%%%%%%%%%
%                       License Page                      %
%%%%%%%%%%%%%%%%%%%%%%%%%%%%%%%%%%%%%%%%%%%%%%%%%%%%%%%%%%%

\restoregeometry

\fancyhead{\includegraphics[width=\linewidth]{assets/UOC_header.png}}
\thispagestyle{UOCfooterpage}
~
\vfill
\begin{minipage}[l]{0.4\paperwidth}
    \includegraphics{assets/license.png}\\
    Esta obra está sujeta a una licencia de Reconocimiento-NoComercial-SinObraDerivada \href{http://creativecommons.org/licenses/by-nc-nd/3.0/es/
        }{3.0 España de Creative Commons}    
\end{minipage}
\pagebreak

%%%%%%%%%%%%%%%%%%%%%%%%%%%%%%%%%%%%%%%%%%%%%%%%%%%%%%%%%%%
%                       Summary Page                      %
%%%%%%%%%%%%%%%%%%%%%%%%%%%%%%%%%%%%%%%%%%%%%%%%%%%%%%%%%%%

\pagenumbering{roman}

\section*{\centering FICHA DEL TRABAJO FINAL}
%    \centering
\begin{longtable}{V{4}r|p{9cm}V{4}}
    \hlineB{4}
    \textbf{Título del trabajo:} & \cellcolor{uocgrey}\textit{\thistitle}\\
    \hline
    \textbf{Nombre del autor:} & \cellcolor{uocgrey}\textit{\thisauthor}\\
    \hline
    \textbf{Nombre del consultor/a:} & \cellcolor{uocgrey}\textit{\thissupervisor}\\
    \hline
    \textbf{Nombre del PRA:} & \cellcolor{uocgrey}\textit{\thisPRA}\\
    \hline
    \textbf{Fecha de entrega:} & \cellcolor{uocgrey}\textit{\duedateshort}\\
    \hline
    \textbf{Titulación o programa:} & \cellcolor{uocgrey}\textit{Máster Universitario en Bioinformática y Bioestadística}\\
    \hline
    \textbf{Área del Trabajo Final} & \cellcolor{uocgrey}\textit{Drug Design and Structural Biology}\\
    \hline
    \textbf{Idioma del trabajo:} & \cellcolor{uocgrey}\textit{Inglés}\\
    \hline
    \textbf{Palabras clave:} & \cellcolor{uocgrey}\textit{Molecular Docking, RNA-binding protein, Allosteric Regulation}\\
    \hline\hline
    \multicolumn{2}{V{4}l V{4}}{\textbf{Resumen del Trabajo}}\\
    \hline
    \multicolumn{2}{V{4}p{14.3cm}V{4}}{\cellcolor{uocgrey}\small{La proteína Musashi-1 (MSI-1) es una proteína de unión a RNA que juega un papel crucial en el desarrollo neuronal tanto en vertebrados como en invertebrados que además también interactúa alostéricamente con ácidos grasos. Estas propiedades hacen que MSI-1 sea una parte biológica interesante en biología sintética para su aplicación como regulador post-transcripcional alostérico de la expresión génética en circuitos généticos sintéticos. Sin embargo, con tal de utilizar con seguridad a MSI-1 como una parte biológica, es necesario entender cómo funcionan sus mecanismos de interacción subyacentes independientemente de los ligandos específicos proporcionados. En este trabajo, se estudian las interacciones alostéricas ocurrentes en MSI-1 mediante simulaciones de docking proteína-RNA y proteína-lípido, a través de una variedad de motivos de RNA sintéticos y de ácidos grasos. Los resultados indican que el comportamiento alostérico de MSI-1 es probablemente debido a un cambio conformacional dentro de la proteína, el cual altera la región de unión a RNA. La eficiencia con la que un ácido graso particular induce este cambio conformacional aún debe ser estudiada, aunque se sugiere que los ácidos grasos más pequeños y flexibles pueden tener una mayor capacidad para inducir cambios conformacionales en la proteína. Los diferentes modelos de interacción proteína-RNA obtenidos fueron inconsistentes para cada motivo RNA, por lo que el efecto de la secuencia nucleotídica del RNA en la interacción continúa sin respuesta.}}\\
    \hline
    \multicolumn{2}{V{4}l V{4}}{\textbf{Abstract}}\\
    \hline
    \multicolumn{2}{V{4}p{14.3cm}V{4}}{\cellcolor{uocgrey}\small{The Musashi-1 (MSI-1) protein is a neural RNA-binding protein that plays a crucial role in neural development in both vertebrates and invertebrates which also interacts allosterically with fatty acids. These properties make MSI-1 an interesting biological part in synthetic biology for its applicability as an allosteric post-transcriptional regulator of gene expression in synthetic gene circuits. However, in order to confidently use MSI-1 as a biological part, there is a need to understand how its underlying interaction mechanisms work irrespective of the particular ligands provided. In this work, the allosteric interactions ocurring in MSI-1 are studied by means of protein-RNA and protein-lipid docking simulations involving a variety of synthetic RNA motifs and several fatty acids. The results indicate that the allosteric behavior of MSI-1 is likely due to a conformational change within the protein that alters the RNA binding region. The efficiency at which a particular fatty acid induces this conformational change is yet to be studied, although it is suggested that smaller and more flexible fatty acid molecules may have a greater ability to induce conformational changes in the protein. The different protein-RNA interaction models obtained were inconsistent within each different RNA motif, so the effect of the RNA nucleic acid sequence itself on the interaction remains unanswered.}
    }\\
    \hlineB{4}
\end{longtable}
\pagebreak

%%%%%%%%%%%%%%%%%%%%%%%%%%%%%%%%%%%%%%%%%%%%%%%%%%%%%%%%%%%
%                     Section Index Page                  %
%%%%%%%%%%%%%%%%%%%%%%%%%%%%%%%%%%%%%%%%%%%%%%%%%%%%%%%%%%%

\tableofcontents

\pagebreak

%%%%%%%%%%%%%%%%%%%%%%%%%%%%%%%%%%%%%%%%%%%%%%%%%%%%%%%%%%%
%                      Figure Index Page                  %
%%%%%%%%%%%%%%%%%%%%%%%%%%%%%%%%%%%%%%%%%%%%%%%%%%%%%%%%%%%

\listoffigures

\pagebreak

%%%%%%%%%%%%%%%%%%%%%%%%%%%%%%%%%%%%%%%%%%%%%%%%%%%%%%%%%%%
%                      Acknowledgements                   %
%%%%%%%%%%%%%%%%%%%%%%%%%%%%%%%%%%%%%%%%%%%%%%%%%%%%%%%%%%%
\section*{Acknowledgements}

I would like to thank my supervisor Emmanuel Fajardo for guiding me during this work. It was always nice to be reassured, even when I thought that the results obtained weren't good enough or when some objectives had to be discarded and the work schedule had to be restructured.\\

I would also like to thank Brian Jiménez García, whom I contacted for the possibility to contribute a tutorial for \texttt{LightDock}. It was fantastic talking to him, as he aided me with some tips and hints so that the \texttt{LightDock} docking simulations worked out better. I hope that we get to collaborate again in the future.\\

Lastly, but no less important, I would like to thank my friends and lab partners María, Raúl and Roser, who were there for me during this difficult period of my life that unfortunately happened to coincide with this Master's Thesis writing. I know things will get better with time, but I doubt I would've made it through this without your unconditional aid. Thank you.

\pagebreak

%%%%%%%%%%%%%%%%%%%%%%%%%%%%%%%%%%%%%%%%%%%%%%%%%%%%%%%%%%%
%                          Sections                       %
%%%%%%%%%%%%%%%%%%%%%%%%%%%%%%%%%%%%%%%%%%%%%%%%%%%%%%%%%%%

% La memoria debería ocupar entre 30 y 60 páginas. Esta cifra es orientativa y no debería limitaros. En cualquier caso, la memoria no puede ocupar más de 90 páginas (sin tener en cuenta los anexos).

\pagenumbering{arabic}

% Section 1: Introduction
\section{Introduction}

\subsection{Context and justification}

The Musashi-1 (MSI-1) protein belongs to a family of neural RNA-binding proteins (RBPs) that play a fundamental role in neural development in both vertebrates and invertebrates \cite{nakamura_1994,sakakibara_1996, good_1998, imai_2001}. In particular, the mouse MSI-1 consists of 362 aminoacids with a molecular mass of 39 kDa \cite{sakakibara_1996}, and contains 2 sequences of around 80 to 90 aminoacids that interact with RNA (RRMs; short for RNA-recognition motifs) following the consensus sequence \texttt{RU}$_n$\texttt{AGU}\footnote{Following the IUPAC standard for degenerate base symbols, \texttt{R} represents both an Adenine (\texttt{A}) and a Guanine (\texttt{G}). Refer to \cite{cornish_1985} for the complete IUPAC standard.}$^,$\footnote{\texttt{U}$_n$ stands for repeating \texttt{U} within the consensus sequence $n$ times. In most cases, $n\in\left\{1,2,3\right\}$ \cite{imai_2001}.} \cite{imai_2001,zearfoss_2014}.\\

Despite multiple efforts, the complete structure of any of the MSI-1 protein homologues remains a mystery. The most notable successes have only been able to completely elucidate MSI-1's RRM1 and RRM2 domains \cite{nagata_1999,miyanoiri_2003,ohyama_2011,lan_2019}, see for example \textbf{Figure \ref{fig:3DRRM1}} for the 3D structure of RRM1. 

\begin{figure}[htbp!]
    \centering
    \includegraphics[width=0.9\linewidth]{assets/RRM1_various.png}
    \caption[3D RRM1 structure of the MSI-1 protein constructred by means of protein crystalization and RMN.]{3D RRM1 structure of the MSI-1 protein constructred by means of protein crystalization and RMN. From left to right: RRM1 in \textit{cartoon} setting, RRM1 in \textit{surface} setting and RRM1 in \textit{surface} setting turned 180$^\circ$. Colors represent the secondary structures found within the protein: cyan corresponds to $\alpha$-helices, magenta corresponds to $\beta$-sheets and salmon corresponds to \textit{random coil}. The model was retrieved from \href{https://www.uniprot.org/uniprotkb/Q61474/entry}{\texttt{Uniprot} entry \texttt{Q61474}}. Visualized through \href{https://pymol.org/2/}{\texttt{Pymol}}.}
    \label{fig:3DRRM1}
\end{figure}

\pagebreak

At the moment, the available complete MSI-1 structures only originate from computational predictions, such as the one shown in \textbf{Figure \ref{fig:3DMSI}} which is an aminoacid-sequence-based prediction yielded by \href{https://alphafold.ebi.ac.uk/}{\texttt{AlphaFold}} \cite{jumper_2021}.

\begin{figure}[htbp!]
    \centering
    \includegraphics[width=0.5\linewidth]{assets/MSI1_complete.png}
    \caption[3D mouse MSI-1 structure as predicted by \texttt{AlphaFold}.]{3D mouse MSI-1 structure as predicted by \href{https://alphafold.ebi.ac.uk/}{\texttt{AlphaFold}}, in \textit{cartoon} setting. Colors represent the secondary structures found within the protein: cyan corresponds to $\alpha$-helices, magenta corresponds to $\beta$-sheets and salmon corresponds to \textit{random coil}. The model was retrieved from \href{https://www.uniprot.org/uniprotkb/Q61474/entry}{\texttt{Uniprot} entry \texttt{Q61474}}. Visualized through \href{https://pymol.org/2/}{\texttt{Pymol}}.}
    \label{fig:3DMSI} 
\end{figure}

With the surge of synthetic biology during the past two decades, RBPs such as MSI-1 have gained a special interest for their applicability as post-transcriptional regulators of gene expression in synthetic gene circuits \cite{belmont_2010,cao_2015,katz_2019}. In particular, Dolcemascolo and colleagues pioneered by designing a synthetic genetic circuit employing MSI-1 as post-transcriptional regulator \cite{dolcemascolo_2022}. In addition, they also exploited MSI-1's particularity of binding to fatty acids in addition to RNA, out of which the cis-9-Octadecenoic acid (also known as oleic acid) presents a higher binding affinity. The latter interaction weakens the MSI-1 protein's RNA-binding affinity, resulting in a dissociation between MSI-1 and the RNA sequence \cite{dolcemascolo_2022,clingman_2014}. In fact, the pocket in which the fatty acids interact with the RMM1 domain of MSI1 can be easily seen in the rightmost protein in \textbf{Figure \ref{fig:3DRRM1}}.\\

Therefore, in order to confidently add MSI-1 as an allosteric regulator to the synthetic biologist's toolbox, there is a need to understand how the underlying interaction mechanisms work. Clingman and colleagues are were the first who performed docking simulations with the MSI-1 core RNA-binding-motif and several fatty acids \cite{clingman_2014}. Nevertheless, these results only hold for the aforementioned RNA-motif: it is expected that any mutation on the RNA-motif would drastically impact the manner in which MSI-1 interacts with the RNA.

\subsection{Objectives}

This Master's Thesis aims to extend the knowledge behind the allosteric interactions occurring in the MSI-1 proteins by performing docking simulations between MSI-1's RRM1 domain, several RNA motifs and fatty acids, with the objective of obtaining a model for each of the combinatorial cases.\\

This general objective breaks down into the following specific objectives:

\begin{enumerate}
    \item \textit{In silico} Generate the 3D structures of 5 mutant MSI-1 RNA-binding motifs used in \cite{dolcemascolo_2022}. The 3D structures of the motifs will depend on the sequence-based-prediction of their secondary structures.
    \item Assess the structure of MSI-1's RRM1-RNA mutant complex through docking for each of the mutants.
    \item Assess the structure of MSI-1's RRM1-fatty acid complex through docking for each of the fatty acids of interest: oleic acid, linoleic acid, palmitoleic acid, arachidonic acid and stearic acid.
    \item Keep the top 2 fatty acids whose interaction with MSI-1 is the clearest.
    \item Assess the structure of RNA mutant-MSI-1's RRM1-fatty acid complex through the coupling of the docking models obtained.
\end{enumerate}
    
\subsection{Impact on sustainability, social-ethical and diversity}

\subsubsection{Sustainability}

The result of this Master's Thesis has had a negligible but negative impact on sustainability. This impact originates on the energy consumption during the following steps:

\begin{enumerate}
    \item Design and debugging of the docking pipeline followed in this work.
    \item Computation of the different models.
\end{enumerate}

This impact can be considered negligible because the net amount of energy consumed during the computations is very small. Nevertheless, it is important to highlight this aspect as there is room for improvement: the source code of the docking software employed in this work is mainly composed of two programming languages: \texttt{C} (72\%) and \texttt{Python} (28\%).\\

Pereira and colleagues developed a benchmark to order programming\linebreak languages with respect of their energy consumption when executing well-known computer science algorithms and data structures \cite{pereira_2017}. Their finding was that the least energy consuming programming language was \texttt{C}, while \texttt{Python} was one of the most energy consuming.\\

It can be discussed that, in order to reach minimal energy consumption, the programming language to be used in software that is going to be distributed and executed multiple times should be exclusively \texttt{C}. The disadvantages of \texttt{C} are that it takes more time to write a full program than with higher level languages (such as \texttt{Python}), it is more difficult to mantain a codebase written in \texttt{C} and that \texttt{C} is not a particularly accessible programming language for non-computer scientists.\\

There is a trade-off that has be to be considered. Let $L$ be a programming language, $E_{W_L}$ be the energy consumed by a computer during the crafting of a software written in $L$, $E_{E_L}$ be the energy consumed by a computer during one execution of the software written in $L$ (in this particular case, one docking simulation with some fixed parameters) and $n$ be the number of times the software is executed, then it is possible to approximate $E_{T_L}(n)$ (the total energy consumed by the software written in $L$ at the moment $n$) as follows:

\begin{equation}
    \label{eq:power}
    E_{T_L}(n)\approx E_{W_L} + n\times E_{E_L}         
\end{equation}

For the pair \texttt{Python} and \texttt{C}, we have that $E_{W_\text{\texttt{Python}}} <<< E_{W_\text{\texttt{C}}}$ and $E_{E_\text{\texttt{Python}}} >>> E_{E_\text{\texttt{C}}}$. So the larger $n$ becomes (the number of times the software is executed), by \textbf{Equation \ref{eq:power}}, \texttt{Python} becomes less and less energy efficient when compared to \texttt{C}.\\

Nevertheless, as mentioned previously, the codebase of the software used is in its majority written in \texttt{C}, the energy consumption of a personal computer is low and the number of executions $n$ in this work is small (less than $10^2$). Therefore this negative impact on energy consumption can be considered negligible.

\subsubsection{Social and ethical responsibility}

The technical nature of this work separates it from any positive nor negative impacts on socio-ethical aspects. There are no conflicts of interests with the results, and this work's material (models and source codes) are freely available to anyone through a public \texttt{GitHub} repository.\\

In addition, this work doesn't have any socio-ethical motivations, as the main objective is to gather more technical knowledge for the field of molecular biology.


\subsubsection{Diversity, gender and human rights}

Again, the technical nature of this work separates it from any positive nor negative impacts on diversity, gender and human rights. The references and materials employed within this work were used without any gender nor race bias.\\

Furhermore, this work doesn't have any diversity, gender nor human rights motivations. However, this work attempts to be accessible to as many people as possible, in the spirit of the universal right for knowledge and education. Moreover, this work exclusively employed license-free and open sourced software so that all the dependencies are accessible to anyone.

\subsection{Approach and methods}

This work aims to extend the work of Dolcemascolo and colleagues \cite{dolcemascolo_2022} where they designed a synthetic genetic circuit that employed MSI-1 as post-\linebreak transcriptional regulator. Thanks to the plasticity of RNA, several mutants of MSI-1's RNA-binding core motif were designed, and it was shown that gene regulation (which is tightly dependent on the MSI-1-RNA interaction) was altered for all mutants: some of them displayed an increase in regulation fold change while others displayed a decrease. In addition, the effects of MSI-1 induced regulation were partially nullified in presence of oleic acid.\\

The initial step of this work will consist in the retrieval of several 3D structures from the appropriate databases:

\begin{itemize}
    \item the 3D structure of mouse MSI-1's RMM1 is retrieved from \href{https://www.uniprot.org/}{\texttt{UniProt}}.
    \item the 3D structures of several fatty acids (oleic acid among them) are retrieved from \href{https://www.chemspider.com/}{\texttt{chemspider}}.
\end{itemize}

Next, some of the RNA sequences employed in \cite{dolcemascolo_2022} are retrieved and their 3D model structures are computed by means of \href{https://nupack.org/}{\texttt{NUPACK}} and \href{https://rnacomposer.cs.put.poznan.pl/}{\texttt{RNAComposer}} web server.\\

Then, all the docking simulations are performed:

\begin{enumerate}
    \item Between each RNA sequence and MSI-1 RMM1.
    \item Between each fatty acid and MSI-1 RMM1.
\end{enumerate}

The choice of docking software will vary for the type of docking simulation to be performed:

\begin{itemize}
    \item For protein-RNA docking, the docking software will be \href{https://lightdock.org/}{\texttt{LightDock}} because it is license-free, open source and it is partially written in \texttt{Python}.
    \item For protein-lipid docking, the docking software will be \href{https://vina.scripps.edu}{\texttt{AutoDock Vina}}, aided by \href{https://ccsb.scripps.edu/mgltools/}{\texttt{AutoDock Tools}} for the preparation step.
\end{itemize}

 The best models will be selected based on the chosen scoring function and biological significancy. In addition, all the models will be visualized with \href{https://pymol.org/2/}{\texttt{Pymol}}.

\subsection{Work plan}

This work is divided in the following tasks (and respective subtasks):

\begin{enumerate}
    \item Perform docking simulations of MSI-1 and a collection of mutants of the MSI-1 RNA-binding-motif.
        \begin{enumerate}
            \item Generate in silico a collection of up to 5 interesting mutant MSI-1 RNA-binding motifs.
            \item Perform docking simulations between MSI-1 and each of the mutants.
            \item Discard those mutants whose interaction with MSI-1 is poor.
        \end{enumerate}
    \item Perform docking simulations of MSI-1 and a collection of fatty acids.
        \begin{enumerate}
            \item Perform docking simulations between MSI-1 and several fatty acids of interest: oleic acid, linoleic acid, palmitoleic acid, arachidonic acid and stearic acid.
            \item Keep the top 2 fatty acids whose resulting model is the best.
        \end{enumerate}
    \item Perform docking simulations of MSI-1 and a subset of mutants of the MSI-1 RNA-binding-motifs in presence of fatty acids.
    \item Coupling of the successful docking models to visualize the lipid-MSI1-RNA complex.
    \pagebreak
    \item Defense preparation
        \begin{enumerate}
            \item Manuscript
            \item Presentation
        \end{enumerate}
\end{enumerate}

The landmarks of this work are distributed among the following continuous assessment tests:

\begin{itemize}
    \item Continuous Assessment Test 1: Thesis Definition and Work Plan.
    \item Continuous Assessment Test 2: Work Development (Phase 1).\\
        Landmarks:
        \begin{itemize}
            \item Generate in silico a collection of up to 5 interesting mutant MSI-1 RNA-binding motifs.
            \item Perform docking simulations between MSI-1 and each of the mutants {\color{red}\textit{(partially)}}.
        \end{itemize}
    \item Continuous Assessment Test 3: Work Development (Phase 2).
        \begin{itemize}
            \item Perform docking simulations between MSI-1 and each of the mutants.
            \item Discard those mutants whose interaction with MSI-1 is poor.
            \item Perform docking simulations between MSI-1 and several fatty acids of interest: oleic acid, linoleic acid, palmitoleic acid, arachidonic acid and stearic acid {\color{red}\textit{(partially)}}.
        \end{itemize}
    \item Continuous Assessment Test 4: Thesis Closure and Presentation.
        \begin{itemize}
            \item Perform docking simulations between MSI-1 and several fatty acids of interest: oleic acid, linoleic acid, palmitoleic acid, arachidonic acid and stearic acid.
            \item Keep the top 2 fatty acids whose resulting model is the best.
            \item Manuscript closure.
            \item Presentation closure.
        \end{itemize}
\end{itemize}

\pagebreak

The work plan is summarized in \textbf{Figure \ref{fig:gnatt}}:

\begin{figure}[htbp!]
    \centering
    \includegraphics[width=\linewidth, trim={1cm 18.5cm 1cm 2cm},clip]{assets/Plan_de_Trabajo_TFM.pdf}
    \caption{Work plan summarized in a Gnatt chart.}
    \label{fig:gnatt}    
\end{figure}

\subsection{Summary of obtained products}

The main products of this work are:

\begin{enumerate}
    \item The current Master's Thesis manuscript.
    \item 3D structures of 5 MSI-1 RNA-motifs used in \cite{dolcemascolo_2022}.
    \item\textit{los modelos o lo que salga}
    \item\textit{los scripts del pipeline}
    \item\href{https://lightdock.org/tutorials/0.9.3/rna_docking}{A step-by-step tutorial on how to perform a simple protein-RNA docking with \texttt{LightDock}}.
\end{enumerate}

Resulting products are available in different repositories:
\begin{enumerate}
    \item The direct products of this work are present in \href{https://github.com/luksgrin/UOC_TFM}{this \texttt{GitHub} repository}.
    \item The step-by-step tutorial on how to perform a simple protein-RNA docking with \texttt{LightDock} is available at \href{https://github.com/lightdock/lightdock.github.io}{the \texttt{LightDock} \texttt{GitHub} repository}, and on the \href{https://lightdock.org/tutorials/}{\texttt{LightDock} tutorials page}.
\end{enumerate}

\subsection{Summary of other sections of this report}

% Breve explicación de los contenidos de cada capítulo y su relación con el proyecto global. 

The remaing sections of this manuscript are:

\begin{itemize}
    \item\textbf{State of the art}
    \item\textbf{Materials and methods}
    \item\textbf{Results}
    \item\textbf{Conclusions and further work}
    \item\textbf{Glossary}
    \item\textbf{References}
    \item\textbf{Appendix}
\end{itemize}
\pagebreak

% Section 2: State of the art
\section{State of the art}

% Estado del arte del tema en cuestión.
% Debería acabar mostrando por qué el trabajo es importante y aporta algo y con las hipótesis del trabajo.

\subsection{Synthetic Biology}
Synthetic Biology is a modern niche of Biology in which the characteristic ``know-how'' from engineering fields is combined with classical Molecular Biology techniques. Guided by physicist Richard Feynman's famous quote ``\textit{what I cannot create, I do not understand}'', synthetic biologists have developed a deep understanding of how cells control and regulate the processes of gene transcription, translation, metabolite biosynthesis and degradation.\\

It is ``synthetic'' in the sense that synthetic biologists engineer novel (absent in nature) genetic constructions with biological ``pieces'', which usually originate from various different organisms, in order to achieve a phenotype in a target organism that lacks said phenotype in nature. Furthermore, this design usually comes with a mathematical model based in physical first principles of how the system should behave.\\

One of the first successful constructions of this type, and most likely the most popular one, is Gardner's toggle switch \cite{gardner_2000} where they engineered a synthetic toggle switch within \textit{E. coli} by using two different promoters that repressed each other, achieving this way bistability (either one promoter is repressed, or the other). Simultaneously, Elowitz' repressilator \cite{elowitz_2000} was published, where they engineered a synthetic ``clock'' within \textit{E. coli} by chaining 3 repressive promoters that repressed one another in chain, in such a way that the genes controlled by them had periodic expression.\\

These constructions, and all the ones that followed, would not have been possible without a (by then, small) repository of genetic parts, which in turn would not have been possible without the arduous work of an innumerable amount of traditional molecular biologists that sequenced and characterized said parts.In fact, the iGEM foundation keeps a repository of standarized biological parts \cite{parts_igem} along with various data such as the assembly methods used, levels of expression, polymerase used, etc...\\

However, there is still a need for more biological parts, especially parts that are ``orthogonal'', i.e. they can co-exist and function correcly within the same organism without deviating from their theoretical behavior. 

% RNA binding proteins
\subsection{RNA-binding proteins}

% Doking
\subsection{Molecular Docking}
\pagebreak

% Section 3: Materials and methods
\section{Materials and methods}

% En estos capítulos, es necesario describir:
% •	los aspectos más relevantes del diseño y desarrollo del trabajo
% •	la metodología elegida para realizar este desarrollo, describiendo las alternativas posibles, las decisiones tomadas, y los criterios utilizados para tomar estas decisiones.
% •	descripción de los productos obtenidos.
 
% La estructuración de los capítulos puede variar en función del tipo de trabajo.  
 
% En caso de que proceda, se incluirá un apartado de “Valoración económica del trabajo”. Este apartado indicará los gastos asociados al desarrollo y mantenimiento del trabajo, así como los beneficios económicos obtenidos y un análisis final sobre la viabilidad del producto.

% The sequence design of the mutant MSI-1 RNA-binding motifs will depend on the mutant’s secondary structure, which will be computed in silico by NUPACK (http://www.nupack.org/). The selected mutants will need to have different yet interesting secondary structures (folded state). • The .mol files of the fatty acids will be retrieved from the Chemspider database (https://www.chemspider.com/). • Docking simulations will be carried out with LightDock (https://lightdock.org/), and the subsequent results will be visualized with the open source version of PyMol (https://github.com/schrodinger/pymol-open-source). • Molecular dynamics simulations will be carried out with AMBER (https://ambermd.org/).

% HITOS Task 1 will be considered a successful if the whole process of design, docking and molecular dynamics simulation is completed for at least 3 different mutants, which will be used in downstream simulations during this work. • Task 2 will be considered a successful if the process of docking and molecular dynamics simulation is completed for at least oleic acid and arachidonic acid, which again will be used in downstream simulations during this work. • Task 3 will be considered successful if at least 6 combinations of molecules are simultaneously docked and their molecular dynamics simulated. This is, 3 different mutants for each of oleic acid and arachidonic acid.

% \subsection{Justificación de los cambios en caso necesario}

% \begin{itemize}
%     \item El objetivo 2 ha sido descartado. Esto se debe a complicaciones adicionales a la hora de hacer el docking proteína-RNA en el programa \texttt{lightdock}:

%     \begin{itemize}
%         \item La librería \texttt{proDy} (una librería que emplea \texttt{lightdock} para las simulaciones de docking) no reconoce cadenas de RNA expecificadas en los ficheros \texttt{.pdb} si estas son especificadas mediante el prefijo \texttt{R} en los identificadores de las bases nucleotídicas (es decir, \texttt{RA}, \texttt{RG}, \texttt{RU} y \texttt{RC}) en la estructura de la molécula. Este hecho resulta en un error en el paso de setup de \texttt{lightdock}.
    
%         La mitigación de este problema consistió en crear un script de \texttt{Python} que elimina ese prefijo \texttt{R} de los ficheros \texttt{.pdb}. Este script se encuentra en \textbf{\nameref{anexo_A}}.
    
%         \item Las simulaciones de docking no funcionaban puesto que el RNA tenía conflictos con los parámetros del AMBER \textit{force field}. Esto se debe a que los parámetros que \texttt{lightdock} debía usar para RNA, se especifican precisamente con el prefijo \texttt{R} eliminado en la mitigación anterior. Por lo tanto, la mitigación de este problema consiste en volver a añadir el prefijo \texttt{R} a los ficheros \texttt{.pdb} mediante un script de \texttt{Python}. Este script se encuentra en \textbf{\nameref{anexo_B}}.
    
%         \item Pese a la mitigación anterior, las simulaciones continuaban fallando. Esto se debía a que los RNAs empleados terminan en ``OH'' en los extremos 3' y 5'. \texttt{lightdock} no reconoce estos átomos como parte de la estructura ni tiene parámetros de AMBER \textit{force field} para ellos. Como mitigación, he decidido eliminar esos átomos mediante un script de \texttt{Python}. Asumo que eliminar un par átomos en los extremos de la molécula de RNA no marca mucha diferencia a la hora de hacer el docking. Este script se encuentra en \textbf{\nameref{anexo_C}}.
    
%         \item Una vez más, pese a las mitigaciones llevadas a cabo anteriormente, las simulaciones seguían fallando. Esta vez, el error venía de parte del fichero \texttt{.pdb} de la proteína: resulta que el aminoácido Histidina (con el identificador \texttt{HIS}) no existe en el AMBER \textit{force field}. Esto se debe a que para el AMBER \textit{force field}, Histidina puede ser uno de 3 residuos \cite{amber_histidine}:

%         \begin{enumerate}
%             \item\texttt{HID}: Histidina con un hidrógeno en el nitrógeno delta
%             \item\texttt{HIE}: Histidina con un hidrógeno en el nitrógeno epsilon
%             \item\texttt{HIP}: Histidina con hidrógenos en ambos nitrógenos (delta y epsilon). Esta Histidina tiene carga positiva
%         \end{enumerate}

%         La mitigación consistió en cambiar el identificador \texttt{HIS} por el identificador adecuado mediante un script de \texttt{Python}. Este script se encuentra en \textbf{\nameref{anexo_D}}.

%         \item Con las mitigaciones anteriores, \texttt{lightdock} ejecutó sin problemas. Sin embargo, \texttt{lgd\_generate\_conformations} no ejecutó correctamente a la hora de construir los modelos de docking en formato \texttt{.pdb}. Esto se debe a que \texttt{lgd\_generate\_conformations}, como en uno de los pasos anteriores, no reconocía la estructura de RNA por el prefijo de la \texttt{R}. Por ende, hubo que añadir un paso de eliminación de ese prefijo de la \texttt{R}. Esto se llevó a cabo mediante el script de \texttt{Python} encontrado en \textbf{\nameref{anexo_A}}.

%     \end{itemize}

%     Con esas mitigaciones, los dockings se ejecutaron sin problema. Resolver las anteriores complicaciones implicaron una mayor inversión de tiempo en la parte de docking de este trabajo. Este hecho, y tras consultarlo con el supervisor de este trabajo, llevó a desechar la parte de dinámica molecular por completo.

%     \item El objetivo 4 no se ha completado puesto que no se había contemplado para este período de tiempo. Las simulaciones de docking con los ácidos grasos están siendo ejecutadas durante la escritura de este informe.

% \end{itemize}

% \section{Relación de las actividades realizadas}

% \subsection{Actividades previstas en el plan de trabajo}

% Las actividades llevadas a cabo son las siguientes:

% \begin{enumerate}
%     \item Creación de los scripts de \texttt{Python} mencionados en las mitigaciones anteriores.

%     \item Ejecución de las simulaciones de docking siguientes:
%     \begin{itemize}
%         \item MSI-1 y el motivo de RNA original
%         \item MSI-1 y el motivo de RNA original (con estructura lineal)
%         \item MSI-1 y el mutante de RNA número 1
%         \item MSI-1 y el mutante de RNA número 2
%         \item MSI-1 y el mutante de RNA número 3
%         \item MSI-1 y el mutante de RNA número 4
%         \item MSI-1 y el mutante de RNA número 5
%     \end{itemize}

%     \item Selección de los 10 modelos de docking con mejor score de luciferina para cada una de las simulaciones mencionadas anteriormente.

%     \item Visualización de los modelos generados por \texttt{lightdock} mediante \texttt{pymol}.

%     Los resultados se encuentran en los siguientes anexos:
%     \begin{itemize}
%         \item\textbf{\nameref{anexo_E}}
%         \item\textbf{\nameref{anexo_F}}
%         \item \textbf{\nameref{anexo_G}}
%         \item \textbf{\nameref{anexo_H}}
%         \item \textbf{\nameref{anexo_I}}
%         \item \textbf{\nameref{anexo_J}}
%         \item \textbf{\nameref{anexo_K}}
%         \item \textbf{\nameref{anexo_L}}
%         \item \textbf{\nameref{anexo_M}}
%         \item \textbf{\nameref{anexo_N}}
%         \item \textbf{\nameref{anexo_O}}
%     \end{itemize}

% \end{enumerate}

% \subsection{Actividades no previstas y realizadas o programadas}

% Adicionalmente, se llevaron las siguientes actividades no previstas:

% \begin{itemize}

%     \item Se llevó a cabo la creación de un script de \texttt{Python} cuyo objetivo es añadir la estructura secundaria de la proteína MSI-1 a los modelos de docking (dado que \texttt{lightdock} elimina esa información).

%     \item Se inició una conversación con el equipo desarrollador de \texttt{lightdock} con el objetivo de elaborar un tutorial para el caso de docking proteína-RNA basado en este trabajo.

% \end{itemize}


% \item El objetivo 2 no se ha cumplido debido a unas complicaciones a la hora de hacer el docking proteína-RNA en el programa \texttt{lightdock}. 
    
% Varios intentos de \textit{setup} y simulación de docking fallaban sin motivo aparente. Tras varios intentos, descubrí que el problema se debía a que la protonación de los ficheros \texttt{.pdb} correspondientes a los motivos de RNA y el fichero \texttt{.pdb} del RRM1 de la proteína MSI-1 eran incompatibles. La mitigación es aparentemente sencilla y consiste en los siguientes pasos:

% \begin{itemize}
%     \item Emplear el software \href{https://github.com/rlabduke/reduce}{\texttt{reduce}} para corregir la protonación en los ficheros \texttt{.pdb}.
%     \item Emplear el software \href{https://github.com/haddocking/pdb-tools/}{\texttt{pdb-tools}} para corregir la numeración de los átomos en los ficheros \texttt{.pdb}.
%     \item Controlar que los protones corregidos por \texttt{reduce} a las estructuras de RNA sean 100\% compatibles con el campo de fuerzas \texttt{AMBER94} (que es el que se emplea como función \texttt{score} cuando se realiza docking con ácidos nucleicos). La documentación de \texttt{lightdock} proporciona un script para llevar a cabo este control: \href{https://lightdock.org/tutorials/0.9.1/dna_docking/data/reduce_to_amber.py}{\texttt{reduce\_to\_amber.py}}.
% \end{itemize}

% En principio, una vez llevados a cabo estos pasos, las simulaciones de docking pueden llevarse a cabo con normalidad. Por lo tanto, es necesario la descarga del software mencionado anteriormente.

% \item El objetivo 3 no se ha completado puesto que no se había contemplado para este período de tiempo. Sin embargo, he tenido que llevar a cabo un cambio fundamental que afecta el cumplimiento de este objetivo. Originalmente, el plan de trabajo contemplaba que las simulaciones de dinámica mulecular se llevasen a cabo con el software \href{https://ambermd.org/}{\texttt{Amber}}. Sin embargo, en contraste con mi estudio inicial durante la elaboración del plan de trabajo, \texttt{Amber} requiere una licencia de pago. Por lo tanto, siguiendo una de las mitigaciones propuestas en el plan de trabajo, las simulaciones de dinámica molecular se llevarán a cabo con \href{http://www.biomolecular-modeling.com/Abalone/}{\texttt{Abalone II}}.

% Las actividades llevadas a cabo son las siguientes:

% \begin{enumerate}
%     \item Diseño de 5 mutantes para el motivo de RNA reconocido por la proteína MSI-1. Para la obtención del motivo de RNA \texttt{wild-type} que interacciona con la proteína MSI-1, se acudió a los trabajos \cite{imai_2001} y \cite{clingman_2014}. Sin embargo, este motivo existe en un contexto de secuencia adicional que permite su correcto plegamiento a una estructura secundaria reconocida por MSI-1. Por ello, siguiendo el ejemplo de \cite{dolcemascolo_2022}, se añadieron una serie de ácidos nucleicos alrededor del motivo de reconocimiento para que el plegamiento predicho de la secuencia de RNA fuese como en las referencias \cite{imai_2001} y \cite{clingman_2014}. La estructura secundaria del plegamiento de la secuencia de RNA fue comprobada mediante \href{http://www.nupack.org/}{\texttt{Nupack}}.
%     \item Mediante la secuencia de RNA resultante, se generaron 5 mutantes mediante sustituciones e inserciones de bases nucleotídicas. Los mutantes candidatos fueron seleccionados según su estructura secundaria, comprobada mediante \texttt{Nupack}. Las secuencias finales se encuentran en el \textbf{\nameref{anexo_A}}.
%     \item Las secuencias mencionadas anteriormente fueron tabuladas, y se añadieron el resultado del alineamiento con la secuencia original y la estructura secundaria (en formato \texttt{dot-bracket}). Esta tabla se encuentra en el \textbf{\nameref{anexo_B}}.
%     \item Los procedimientos anteriores se llevaron a cabo mediante un script de \texttt{Python}, el cual se encuentra en el \textbf{\nameref{anexo_C}}.
%     \item Se computó la conformación 3D de cada uno de los motivos de RNA (gracias a su secuencia nucleotídica y estructura secundaria) mediante el servidor web \href{https://rnacomposer.cs.put.poznan.pl/}{\texttt{RNAComposer}}. Las estructuras 3D de los RNAs se encuentran en el \textbf{\nameref{anexo_D}}.
%     \item Se descargó el \href{https://alphafold.ebi.ac.uk/files/AF-Q61474-F1-model_v4.pdb}{modelo 3D de la proteína MSI-1 de ratón} desde \href{https://www.uniprot.org/}{Uniprot}. Este modelo 3D es el predicho por \texttt{AlphaFold}, y muestra muchas zonas sin estructura (o \textit{random coil}). Esto puede ser indicativo de una mala predicción de parte de \texttt{AlphaFold}, por ende también se descargó el \href{https://www.ebi.ac.uk/pdbe/entry-files/download/pdb1uaw.ent}{modelo 3D de resonancia magnética nuclear de experimentos de cristalografía de proteínas del motivo RRM1 de MSI-1}. En función de los resultados de los simulaciones de docking, se puede considerar desechar el uso de la proteína entera y centrarse únicamente en el motivo RRM1. Las estructuras 3D de la predicción de \texttt{AlphaFold} de MSI-1 y el RRM1 se encuentran en \textbf{\nameref{anexo_E}} y \textbf{\nameref{anexo_F}}.
%     \item Se llevaron a cabo varios intentos fallidos de simulaciones de docking entre el motivo RRM1 y el motivo de RNA denominado \texttt{orig}. En dichas simulaciones fallidas se aplicaron distintos números de \textit{gloworms}, \textit{swarms} y \textit{steps}.
%     \item Se descargaron desde \href{https://www.chemspider.com/}{chemspider} los ficheros \texttt{.mol} de los siguientes ácidos grasos: \href{https://www.chemspider.com/Chemical-Structure.393217.html}{ácido oleico}, \href{https://www.chemspider.com/Chemical-Structure.4444105.html}{ácido linoleico}, \href{https://www.chemspider.com/Chemical-Structure.392692.html}{ácido araquidónico}, \href{https://www.chemspider.com/Chemical-Structure.960.html}{ácido palmítico} y \href{https://www.chemspider.com/Chemical-Structure.5091.html}{ácido esteárico}. Estas moléculas se emplearán simulaciones de docking y dinámica molecular futuros. Las estructuras 3D de los ácidos grasos se encuentran en \textbf{\nameref{anexo_G}}.
% \end{enumerate}

% \subsection{Actividades no previstas y realizadas o programadas}

% No se tenía previsto la necesidad de programas adicionales como part del tratamiento de los ficheros \texttt{.pdb} previos a las simulaciones de docking para el caso proteína-RNA. Por ende, se tiene previsto llevar a cabo las siguientes actividades:

% \begin{enumerate}
%     \item Emplear el software \href{https://github.com/rlabduke/reduce}{\texttt{reduce}} para corregir la protonación de las moléculas.
%     \item Emplear el software \href{https://github.com/haddocking/pdb-tools/}{\texttt{pdb-tools}} para corregir la numeración de átomos.
% \end{enumerate}
\pagebreak

% Section 4: Results
\section{Results}

% Detallad en este apartado los resultados obtenidos utilizando la metodología descrita en el apartado anterior.
\pagebreak

% Section 5: Conclusions and further work
\section{Conclusions and further work}


% Este capítulo debe incluir:
% •	Una descripción de las conclusiones del trabajo:
% o	¿Una vez se han obtenido los resultados qué conclusiones se extrae?
% o	¿Estos resultados son los esperados? ¿O han sido sorprendentes? ¿Por qué?
% •	Una reflexión crítica sobre la consecución de los objetivos planteados inicialmente:
% o	¿Hemos alcanzado todos los objetivos? Si la respuesta es negativa, ¿por qué?
% •	Un análisis crítico del seguimiento de la planificación y metodología a lo largo del producto:
% o	¿Se ha seguido la planificación?
% o	¿La metodología prevista ha sido suficientemente adecuada?
% o	¿Ha sido necesario introducir cambios para garantizar el éxito del trabajo? ¿Por qué?
% •	De los impactos previstos en 1.3 (ético-sociales, de sostenibilidad y de diversidad), evaluar/mencionar si se han mitigado (si eran negativos) o si se han logrado (si eran positivos). 
% •	Si han aparecido impactos no previstos en 1.3, evaluar/mencionar cómo se han mitigado (si eran negativos) o qué han aportado (si eran positivos).
% •	Las líneas de trabajo futuro que no han podido explorarse en este trabajo y han quedado pendientes.

\pagebreak

% Section 6: Glossary
\section{Glossary}
% Definición de los términos y acrónimos más relevantes utilizados en la Memoria.

\begin{itemize}
    \item[]\textbf{MSI-1}
    \item[]\textbf{RBP}
    \item[]\textbf{RRM}
    \item[]\textbf{NMR}
\end{itemize}
\pagebreak

% Section 7: References
\bibliographystyle{ieeetr}
\bibliography{references.bib}
% Lista numerada de las referencias bibliográficas utilizadas en la memoria. En cada lugar donde se utilice una referencia dentro del texto, debe indicarse citando el número de la referencia, por ejemplo: [7].
 
% Es muy importante incluir todas las referencias utilizadas y citarlas apropiadamente, es decir, incluyendo toda la información necesaria para identificar la referencia. La información mínima a incluir según el tipo de referencia es:
 
% •	Libro: Autores, Título, Edición (en su caso) Editorial, Ciudad, Año.
% •	Artículo de revista: Autores, Título, Nombre de la Revista, Número de Página inicial y final, Número de la revista / Volumen, Año.
% •	Web: URL y fecha en la que se ha visitado.	

\pagebreak

% Section 8: Appendix
\setcounter{page}{1}
\pagenumbering{Roman}

\section{Appendix}

% Listado de apartados que son demasiado extensos para incluir en la memoria y tienen un carácter autocontenido (por ejemplo, manuales de usuario, manuales de instalación, etc.)
 
% Dependiendo del tipo de trabajo, es posible que no sea necesario añadir ningún anexo.

\subsection*{Appendix A: RNA motifs employed in this work, as seen in \cite{dolcemascolo_2022} (in \texttt{.fasta} format)}\label{appendix_A}
\addcontentsline{toc}{subsection}{Appendix A: RNA motifs employed in this work, as seen in \cite{dolcemascolo_2022} (in \texttt{.fasta} format)}

\lstinputlisting{assets/motifs.fasta}

\subsection*{Appendix B: \texttt{Python} script to treat the RNA motifs}\label{appendix_B}
\addcontentsline{toc}{subsection}{Appendix B: \texttt{Python} script to treat the RNA motifs}

\lstinputlisting[language=Python]{assets/motifs_treatment.py}

\subsection*{Appendix C: \texttt{Python} script to relabel \texttt{HIS} residues for their corresponding \texttt{AMBER94} compatible residues}\label{appendix_C}
\addcontentsline{toc}{subsection}{Appendix C: \texttt{Python} script to relabel \texttt{HIS} residues for their corresponding \texttt{AMBER94} compatible residues}\label{appendix_C}

\lstinputlisting[language=Python]{assets/fixHIS.py}

\subsection*{Appendix D: \texttt{Python} script to remove 3' and 5' hydroxile groups in the RNA structures}\label{appendix_D}
\addcontentsline{toc}{subsection}{Appendix D: \texttt{Python} to remove 3' and 5' hydroxile groups in the RNA structures}

\lstinputlisting[language=Python]{assets/removeEnds.py}

\subsection*{Appendix E: \texttt{Python} script to rename and remove incompatible atom types from the RNA srtructure}\label{appendix_E}
\addcontentsline{toc}{subsection}{Appendix E: \texttt{Python} script to rename and remove incompatible atom types from the RNA srtructure}

\lstinputlisting[language=Python]{assets/reduce_to_amber.py}

\subsection*{Appendix F: \texttt{Python} script to remove the incompatible \texttt{R} tags from the RNA structures}\label{appendix_F}
\addcontentsline{toc}{subsection}{Appendix F: \texttt{Python} script to remove the incompatible \texttt{R} tags from the RNA structures}

\lstinputlisting[language=Python]{assets/manipulateRNA.py}

\subsection*{Appendix G: \texttt{Python} script to recover the \texttt{R} tags from the RNA structures}\label{appendix_G}
\addcontentsline{toc}{subsection}{Appendix G: \texttt{Python} script to recover the \texttt{R} tags from the RNA structures}

\lstinputlisting[language=Python]{assets/manipulateRNAagain.py}

\subsection*{Appendix H: \texttt{Bash} script to cluster and rank the models generated by \texttt{LightDock}}\label{appendix_H}
\addcontentsline{toc}{subsection}{Appendix H: \texttt{Bash} script to cluster and rank the models generated by \texttt{LightDock}}

\lstinputlisting[language=Python]{assets/clusterrank.sh}

\subsection*{Appendix I: \texttt{Bash} script to run a complete protein-RNA docking simulation with \texttt{LightDock}}\label{appendix_I}
\addcontentsline{toc}{subsection}{Appendix I: \texttt{Bash} script to run a complete protein-RNA docking simulation with \texttt{LightDock}}

\lstinputlisting[language=Python]{assets/run1.sh}

% \subsection*{Anexo E: Visualización de los 10 mejores modelos de docking entre MSI-1 y el RNA original (\textit{wild type})}\label{anexo_E}
% \addcontentsline{toc}{subsection}{Anexo E: Visualización de los 10 mejores modelos de docking entre MSI-1 y el RNA original (\textit{wild type})}

% \begin{center}
%     \includegraphics[width=\linewidth]{assets/RMM1_orig_ALL.png}
% \end{center}

% Podemos observar que los modelos difieren entre sí. En los subsiguientes anexos, estos modelos serán desglosados. Visualizado con \href{https://pymol.org/2/}{\texttt{Pymol}}.

% \subsection*{Anexo F: Visualización del mejor modelo de docking entre MSI-1 y el RNA original (\textit{wild type})}\label{anexo_F}
% \addcontentsline{toc}{subsection}{Anexo F: Visualización del mejor modelo de docking entre MSI-1 y el RNA original (\textit{wild type})}

% \begin{center}
%     \includegraphics[width=\linewidth]{assets/RMM1_orig_top0.png}
% \end{center}

% Visualizado con \href{https://pymol.org/2/}{\texttt{Pymol}}.

% \subsection*{Anexo G: Visualización del segundo mejor modelo de docking entre MSI-1 y el RNA original (\textit{wild type})}\label{anexo_G}
% \addcontentsline{toc}{subsection}{Anexo G: Visualización del segundo mejor modelo de docking entre MSI-1 y el RNA original (\textit{wild type})}

% \begin{center}
%     \includegraphics[width=\linewidth]{assets/RMM1_orig_top1.png}
% \end{center}

% Visualizado con \href{https://pymol.org/2/}{\texttt{Pymol}}.

% \subsection*{Anexo H: Visualización del tercer al décimo mejores modelos de docking entre MSI-1 y el RNA original (\textit{wild type})}\label{anexo_H}
% \addcontentsline{toc}{subsection}{Anexo H: Visualización del tercer al décimo mejores modelos de docking entre MSI-1 y el RNA original (\textit{wild type})}

% \begin{center}
%     \includegraphics[width=\linewidth]{assets/RMM1_orig_2to8.png}
% \end{center}

% Podemos observar que estos modelos son casi equivalentes. Visualizado con \href{https://pymol.org/2/}{\texttt{Pymol}}.

% \subsection*{Anexo I: Visualización de los 10 mejores modelos de docking entre MSI-1 y el RNA mutante 1}\label{anexo_I}
% \addcontentsline{toc}{subsection}{Anexo I: Visualización de los 10 mejores modelos de docking entre MSI-1 y el RNA mutante 1}

% \begin{center}
%     \includegraphics[width=0.9\linewidth]{assets/RMM1_mut1_ALL.png}
% \end{center}

% Podemos observar que los modelos difieren muchísimo entre sí. Parece ser que la mutación desestabiliza la interacción.  Visualizado con \href{https://pymol.org/2/}{\texttt{Pymol}}.

% \subsection*{Anexo J: Visualización de los 10 mejores modelos de docking entre MSI-1 y el RNA mutante 2}\label{anexo_J}
% \addcontentsline{toc}{subsection}{Anexo J: Visualización de los 10 mejores modelos de docking entre MSI-1 y el RNA mutante 2}

% \begin{center}
%     \includegraphics[width=0.9\linewidth]{assets/RMM1_mut2_ALL.png}
% \end{center}

% Podemos observar que los modelos difieren muchísimo entre sí. Parece ser que la mutación desestabiliza la interacción.  Visualizado con \href{https://pymol.org/2/}{\texttt{Pymol}}.

% \subsection*{Anexo K: Visualización de los 10 mejores modelos de docking entre MSI-1 y el RNA mutante 3}\label{anexo_K}
% \addcontentsline{toc}{subsection}{Anexo K: Visualización de los 10 mejores modelos de docking entre MSI-1 y el RNA mutante 3}

% \begin{center}
%     \includegraphics[width=0.9\linewidth]{assets/RMM1_mut3_ALL.png}
% \end{center}

% Podemos observar que los modelos difieren muchísimo entre sí. Parece ser que la mutación desestabiliza la interacción.  Visualizado con \href{https://pymol.org/2/}{\texttt{Pymol}}.

% \subsection*{Anexo L: Visualización de los 10 mejores modelos de docking entre MSI-1 y el RNA mutante 4}\label{anexo_L}
% \addcontentsline{toc}{subsection}{Anexo L: Visualización de los 10 mejores modelos de docking entre MSI-1 y el RNA mutante 4}

% \begin{center}
%     \includegraphics[width=0.9\linewidth]{assets/RMM1_mut4_ALL.png}
% \end{center}

% Podemos observar que los modelos difieren muchísimo entre sí. Parece ser que la mutación desestabiliza la interacción.  Visualizado con \href{https://pymol.org/2/}{\texttt{Pymol}}.

% \subsection*{Anexo M: Visualización de los 10 mejores modelos de docking entre MSI-1 y el RNA mutante 5}\label{anexo_M}
% \addcontentsline{toc}{subsection}{Anexo M: Visualización de los 10 mejores modelos de docking entre MSI-1 y el RNA mutante 5}

% \begin{center}
%     \includegraphics[width=0.9\linewidth]{assets/RMM1_mut5_ALL.png}
% \end{center}

% Podemos observar que los modelos difieren muchísimo entre sí. Parece ser que la mutación desestabiliza la interacción.  Visualizado con \href{https://pymol.org/2/}{\texttt{Pymol}}.

% \subsection*{Anexo N: Tabla de scores para los 10 mejores modelos de cada experimento de docking}\label{anexo_N}
% \addcontentsline{toc}{subsection}{Anexo N: Tabla de scores para los 10 mejores modelos de cada experimento de docking}

% \lstinputlisting{assets/allScores.table}

% \subsection*{Anexo O: Visualización de los 10 mejores modelos de docking entre MSI-1 y el RNA original en estructura lineal (sin estructura secundaria)}\label{anexo_O}
% \addcontentsline{toc}{subsection}{Anexo O: Visualización de los 10 mejores modelos de docking entre MSI-1 y el RNA original en estructura lineal (sin estructura secundaria)}

% \begin{center}
%     \includegraphics[width=\linewidth]{assets/linear_no_restraints.png}
% \end{center}

% Podemos observar que los modelos difieren muchísimo entre sí. En comparación con la simulación mediante el RNA con la estructura secundaria adecuada, la estructura lineal da peores resultados. Este resultado es coherente con la naturaleza, puesto que en disolución, el RNA adopta estructuras secundarias para la minimización de su energía libre de Gibbs.  Visualizado con \href{https://pymol.org/2/}{\texttt{Pymol}}.


\end{document}